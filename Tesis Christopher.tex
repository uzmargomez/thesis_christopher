\documentclass[11pt,twoside,openright,spanish]{report}
%\documentclass[runningheads,a4paper]{llncs}
%\documentclass{article}
\usepackage{array}
\newcolumntype{L}{>{\centering\arraybackslash}m{3cm}}
\usepackage{booktabs}
\usepackage[utf8]{inputenc}
\usepackage[spanish,es-tabla,mexico]{babel}
\usepackage{UNAMThesis}
\usepackage{titlesec, blindtext, color}
\usepackage{amsmath}
\usepackage{amsfonts}
\usepackage{fancyhdr}   
\usepackage{tensor}
\usepackage{multirow} 
\usepackage{tabu}
\usepackage[style]{fncychap}
\usepackage{tabularx}
\usepackage{makeidx}
\usepackage{emptypage}
\usepackage{floatrow}
%\usepackage{\tiny }{mathrsfs}
\usepackage[font=footnotesize,labelfont=bf]{caption} % Pie de figura
\usepackage{tabulary}
\usepackage[top=0.8in]{geometry} % Margen superior más adecuado
\usepackage{makeidx}
\usepackage{verbatim}
\usepackage{float} % Permite poner las imágenes en donde queramos
\usepackage{amssymb}
\usepackage{subfig}
\usepackage{layout}
\usepackage{calligra} 
\usepackage[dvipsnames,table,xcdraw]{xcolor}
\usepackage{graphicx} % Nos permite utilizar imágenes.
\usepackage{graphics}
\usepackage{caption}
\usepackage{mwe}
\usepackage{hyperref}
\usepackage{framed}
\usepackage{leftidx}
\usepackage{dsfont}
\usepackage{mathtools}
\usepackage{enumerate}
\usepackage{chngcntr} % Para no resetear notas de pie de página
\usepackage{setspace}
\usepackage{epstopdf}
\usepackage{setspace}
\usepackage{booktabs}
\usepackage{url}
\usepackage{calc}  
\usepackage{calrsfs}
\usepackage{enumitem}
\usepackage[T1]{fontenc}
\usepackage[bitstream-charter]{mathdesign}
\usepackage{etoolbox}
% code listing settings
\usepackage{listings}
\usepackage{wrapfig,lipsum,booktabs}
\usepackage[customcolors,shade]{hf-tikz}
\usepackage[authoryear,round]{natbib}


 

\DeclareFloatVCode{myrowsep}{\vskip 4ex}

\newcounter{bibcount}
\makeatletter
\patchcmd{\@lbibitem}{\item[}{\item[\hfil\stepcounter{bibcount}{\thebibcount.}}{}{}
\setlength{\bibhang}{2\parindent}
\renewcommand\NAT@bibsetup%
[1]{\setlength{\leftmargin}{\bibhang}\setlength{\itemindent}{-\parindent}%
	\setlength{\itemsep}{\bibsep}\setlength{\parsep}{\z@}}
\makeatother


\bibliographystyle{apalike2mod}

\hfsetbordercolor{blue}

\captionsetup[table]{position=bottom}

\apptocmd{\thebibliography}{\csname cleardoublepage phantomsection\endcsname\addcontentsline{toc} {chapter}{Bibliografía}}{}{}

\makeatletter
\patchcmd{\ttlh@hang}{\parindent\z@}{\parindent\z@\leavevmode}{}{}
\patchcmd{\ttlh@hang}{ }{}{}{}
\makeatother

\numberwithin{equation}{chapter}
\numberwithin{figure}{chapter}
\numberwithin{table}{chapter}
\counterwithout{footnote}{chapter}

\renewcommand{\arraystretch}{1.2}

%Interlineado
\renewcommand{\baselinestretch}{1.5}

\setlength{\headheight}{15pt} 

\logounam{Imagenes/Escudo-UNAM}
\logoinstitute{Imagenes/Escudo-IBT}
\pagenumbering{roman}
\flushbottom
\newtheorem{theorem}{Theorem}
\newtheorem{acknowledgement}[theorem]{Acknowledgement}
\newtheorem{algorithm}[theorem]{Algorithm}
\newtheorem{axiom}[theorem]{Axiom}
\newtheorem{case}[theorem]{Case}
\newtheorem{claim}[theorem]{Claim}
\newtheorem{conclusion}[theorem]{Conclusion}
\newtheorem{condition}[theorem]{Condition}
\newtheorem{conjecture}[theorem]{Conjecture}
\newtheorem{corollary}[theorem]{Corollary}
\newtheorem{criterion}[theorem]{Criterion}
\newtheorem{definition}[theorem]{Definition}
\newtheorem{example}[theorem]{Example}
\newtheorem{exercise}[theorem]{Exercise}
\newtheorem{lemma}[theorem]{Lemma}
\newtheorem{notat}[theorem]{Notat}
\newtheorem{problem}[theorem]{Problem}
\newtheorem{proposition}[theorem]{Proposition}
\newtheorem{remark}[theorem]{Remark}
\newtheorem{solution}[theorem]{Solution}
\newtheorem{summary}[theorem]{Summary}
\newenvironment{proof}[1][Proof]{\textbf{#1.} }{\ \rule{0.5em}{0.5em}}

\newcommand{\grad}{\hspace{-2mm}$\phantom{a}^{\circ}$}



\renewcommand{\sin}{\operatorname{\sen}}

\lfoot[]{}
\cfoot[]{}
\rfoot[]{}
\renewcommand{\footrulewidth}{0pt}
\renewcommand{\headrulewidth}{0.1pt}

%%%%%%%%%%%%%%%%%%%% Reemplazamos l con elle %%%%%%%%%%%%%%%%%%%%%%
\mathcode`l="8000
\begingroup
\makeatletter
\lccode`\~=`\l
\DeclareMathSymbol{\lsb@l}{\mathalpha}{letters}{`l}
\lowercase{\gdef~{\ifnum\the\mathgroup=\m@ne \ell \else \lsb@l \fi}}%
\endgroup

%%%%%%%%%%%%%%%%%%%%% Norma y valor absoluto %%%%%%%%%%%%%%%%%%%%%%
\DeclarePairedDelimiter\abs{\lvert}{\rvert}%
\DeclarePairedDelimiter\norm{\lVert}{\rVert}%

% Swap the definition of \abs* and \norm*, so that \abs
% and \norm resizes the size of the brackets, and the 
% starred version does not.
\makeatletter
\let\oldabs\abs
\def\abs{\@ifstar{\oldabs}{\oldabs*}}
\let\oldnorm\norm
\def\norm{\@ifstar{\oldnorm}{\oldnorm*}}
%%%%%%%%%%%%%%%%%%%%%%%%%%%%%%%%%%%%%%%%%%%%%%%%%%%%%%%%%%%%%%%%%%%
% Change Colors https://en.wikibooks.org/wiki/LaTeX/Colors
\hypersetup{
	bookmarks=true,         % show bookmarks bar?
	unicode=true,          % non-Latin characters in Acrobat’s bookmarks
	pdftoolbar=true,        % show Acrobat’s toolbar?
	pdfmenubar=true,        % show Acrobat’s menu?
	pdffitwindow=false,     % window fit to page when opened
	pdfstartview={FitH},    % fits the width of the page to the window
	pdftitle={Tesis Uzmar},    % title
	pdfauthor={Uzmar Gómez},     % author
	pdfsubject={Subject},   % subject of the document
	pdfcreator={Uzmar Gómez},   % creator of the document
	pdfproducer={Uzmar Gómez}, % producer of the document
	pdfkeywords={}, % list of keywords
	pdfnewwindow=true,      % links in new PDF window
	colorlinks=true,       % false: boxed links; true: colored links
	linkcolor=BlueViolet,          % color of internal links (change box color with linkbordercolor)
	citecolor=BrickRed,        % color of links to bibliography
	filecolor=NavyBlue,      % color of file links
	urlcolor=NavyBlue           % color of external links
}


\newenvironment{changemargin}[3]{
	\begin{list}{}{
			\setlength{\topsep}{#3}
			\setlength{\leftmargin}{#1}
			\setlength{\rightmargin}{#2}
			\setlength{\listparindent}{\parindent}
			\setlength{\itemindent}{\parindent}
			\setlength{\parsep}{\parskip}
		}
		\item[]}{\end{list}}	
	
\usepackage{listings}
\usepackage{color}

\definecolor{dkgreen}{rgb}{0,0.6,0}
\definecolor{gray}{rgb}{0.5,0.5,0.5}
\definecolor{mauve}{rgb}{0.58,0,0.82}

\lstset{frame=tb,
	language=R,
	aboveskip=3mm,
	belowskip=3mm,
	showstringspaces=false,
	columns=flexible,
	basicstyle={\small\ttfamily},
	numbers=left,
	numberstyle=\tiny\color{gray},
	keywordstyle=\color{blue},
	commentstyle=\color{dkgreen},
	stringstyle=\color{mauve},
	breaklines=true,
	breakatwhitespace=true,
	tabsize=4,
	literate={á}{{\'a}}1 {é}{{\'e}}1 {í}{{\'i}}1 {ó}{{\'o}}1 {ú}{{\'u}}1 {Á}{{\'A}}1 {É}{{\'E}}1 {Í}{{\'I}}1 {Ó}{{\'O}}1 {Ú}{{\'U}}1
}

\begin{document}
	
	\renewcommand{\baselinestretch}{1}
	
	\graphicspath{{./Imagenes/}}
	
	\title{Reporte Experiencia Profesional: Metodología y Cálculo de la Reserva de Riesgos en Curso de una Compañía Aseguradora}
	\author{Christopher Gómez Yáñez}
	\institute{Facultad de Ciencias}
	\degree{Actuario}
	\supervisor{Mtro. Alfonso Parrao Guzmán}
	\city{Ciudad Universitaria, CD. MX.}
	\degreemonth{Abril}
	\degreeyear{2020}
	\maketitle
	
	\newpage
	$\ $
	\thispagestyle{empty} % para que no se numere esta pagina
	
	\begin{changemargin}{1cm}{0cm}{1cm}
		
		\vspace{30cm} 
		\begin{center}
			\textit{\textbf{\Large JURADO ASIGNADO}}
		\end{center}
		\vspace{1cm}
		 
		\begin{description}
			\item[]\textbf{Datos del alumno:}\\
			Gómez Yáñez, Christopher\\
			\textit{Número de cuenta:} 307228305\\
			\textit{Carrera:} Actuaria\\
			\textit{Correo:} chris.gomez@ciencias.unam.mx\\
			\textit{Teléfono:} 5532358625
			
			\item[]\textbf{Presidente:}\\
			Dr. Núñez Zúñiga, Darío\\
			\textit{Correo:} nunez@nucleares.unam.mx\\
			\textit{Institución de adscripción:} Instituto de Ciencias Nucleares, UNAM
			\item[]\textbf{Vocal:}\\
			Dr. Matos Chassin, Tonatiuh\\
			\textit{Correo:} tmatos@fis.cinvestav.mx \\
			\textit{Institución de adscripción:} Departamento de Física, CINVESTAV
			\item[]\textbf{Secretario:}\\
			Dr. Alcubierre Moya, Miguel\\
			\textit{Correo:} malcubi@nucleares.unam.mx\\
			\textit{Institución de adscripción:} Instituto de Ciencias Nucleares, UNAM
			\item[]\textbf{1\textsuperscript{er} Suplente:}\\
			Dr. Tejeda Rodríguez, Emilio\\
			\textit{Correo:} etejeda@astro.unam.mx\\
			\textit{Institución de adscripción:} Instituto de Astronomía, UNAM
			\item[]\textbf{2\textsuperscript{do} Suplente:}\\
			Dr. Degollado Daza, Juan Carlos\\
			\textit{Correo:} jcdegollado@ciencias.unam.mx\\
			\textit{Institución de adscripción:} Instituto de Ciencias Físicas, UNAM
		\end{description}
		\thispagestyle{empty}
	\end{changemargin}
	
	
	%\ChNameVar{\bfseries\Large\sf} \ChNumVar{\Huge} \ChTitleVar{\bfseries\Large\rm}
	%\ChRuleWidth{1pt} \ChNameUpperCase \ChTitleUpperCase
	\ChNumVar{\fontsize{50}{50}\usefont{T1}{ptm}{m}{sl}\selectfont}
	\ChTitleVar{\raggedright\Large\sffamily\bfseries}
	
	\evensidemargin 0in 
	\oddsidemargin 0.6in
	
	\newpage{\ } 
	\thispagestyle{empty}
	
	\begin{dedication}
		{\Large{\sffamily{El continuo esfuerzo, no la fortaleza o inteligencia, es la clave para desbloquear nuestro potencial.}}}\\
		\vspace{0.5cm}
		{\normalsize{\bfseries{Winston S. Churchill}}}
	\end{dedication}
	
	\newpage
	$\ $
	\thispagestyle{empty} % para que no se numere esta pagina
	
	\begin{acknowledgements}
	\pagenumbering{Roman}
	 
	A mi mamá. d
	\\

	A XXXXXXXXXXX.
	\\
	
	A XXXXXXXXXXXXXXXXXXXXXX.
	\\
	
	AXXXXXXXXXXXXXXXXXXXXX.
	\\
	
	A XXXXXXXXXXXXXXXXXXXXXX.
	\\
	
	A XXXXXXXXxxXXXXXXXXXXx.
	\\
	
	A XXXXXXXxxxxxxxxxxxxx.   
	\\
	
	A XXXXXXxxxxxxxxXXXXXXXXXXXXXX
	\\
	
	A XXXXXXXXXXXXXXXXx
	\\
	
	A XXXXXXXXXXXXXXXXXXXXXXXXx
	\\
	
		
	\end{acknowledgements}
	
	
	\tableofcontents
	
	
	\addtolength{\headheight}{\baselineskip}
	\fancyhead[LE]{\scshape\thepage\hspace{1cm}\footnotesize\nouppercase{Universidad Nacional Autónoma de México}}
	\fancyhead[RO]{\scshape\footnotesize\nouppercase{Facultad de Ciencias}\hspace{1cm}\normalsize\thepage}
	\pagestyle{fancy}
	\cleardoublepage
		
	\begin{preface}
	\pagenumbering{arabic}
	\addcontentsline{toc}{chapter}{\numberline{}Prefacio}%	
	\doublespacing
	
	En este trabajo se tiene como objetivo describir brevemente las labores de un Actuario dentro de una Compañía Aseguradora para posteriormente abordar una de las partes más escenciales del puesto, la cual es la valuación de la Reserva de Riesgos en Curso. Tambien se mencionaran las herramientas y metodología necesarias para el proceso y las leyes y organismos gubernamentales que dictan las condiciones bajo las cuales debe realizarse el cálculo.
		
	\end{preface}
	
	%--------------------------------------------------------------------------------------------------------------- %
	
	\fancypagestyle{plain}{
		\fancyhead[L]{}
		\fancyhead[C]{}
		\fancyhead[R]{}
		
		\fancyfoot[L]{}
		\fancyfoot[C]{\thepage}
		\fancyfoot[R]{}
		\renewcommand{\headrulewidth}{0pt}
		\renewcommand{\footrulewidth}{0pt}
	}
	
	\fancyhead[LE]{\scshape\thepage\hspace{1cm}\footnotesize\nouppercase{\leftmark}}
	\fancyhead[RO]{\scshape\footnotesize\nouppercase{Sección \rightmark\hspace{1cm}}\normalsize\thepage}
	\fancyhead[LO]{}
	\fancyhead[RE]{}
	\pagestyle{fancy}
	\cleardoublepage
	
	\chapter{Introducción}\label{cap:Introducción}

A continuación se mencionan brevemente las actividades relacionadas con el puesto de Actuario en la una Compañía Aseguradora.

\begin{singlespace}
	\begin{enumerate}
		\item \textbf{Actividades Mensuales}
		\begin{enumerate}
			\item \textit{Procesos del Cierre de Mes} \begin{enumerate}
				 	\item Revisión de bases de información de Cierre:
				 	\\ \-\hspace{0.5cm} El Actuario debe asegurarse de que la información correspondiente a los Asegurados y Siniestros registrados en las bases de datos de la Compañía sea veraz, oportuna, integra y consistente, con el fin de utilizarla en procesos consiguientes, como son reportes y especialmente para el cálculo de la Reserva Técnica de la Compañía.\\
				  \end{enumerate}
			\item \textit{ Cálculo de Reserva Técnica}  	\begin{enumerate}
				\item Verificar Factores de Mercado publicados en la CUSF para calculo de Reservas mediante Metodo Estatutario:\\ \-\hspace{0.5cm}
				Para el Cálculo de Reservas de Subramos cuya información no es suficiente se debe emplear el metodo Estatutario aprovado por la CNSF, para el cual son necesarios Factores obtenidos con la Información de Mercado de Seguros. Ya que el Mercado es cambiante es responsabilidad del Actuario asegurarse que se estan empleando los factores más actualizados al momento del cálculo.
				\item Extracción de información de las Bases de Datos validadas:\\ \-\hspace{0.5cm}
				Para calcular la Reserva es necesario contar con las cifras correspondientes a Asegurados en Vigor y Siniestros Autorizados, para la obtención de la Reserva de Gastos de Administración y la Reserva de Riesgos en Curso, respectivamente.				
				\item Simulación Estocástica en R:\\ \-\hspace{0.5cm}
				Una vez que se cuenta con la información de Siniestralidad Autorizada se generara a partir de élla las Matrices de Siniestralidad sobre las cuales se aplicara el método expuesto en este trabajo, en específico las $n$ simulaciones.
				\item Obtención de los resultados del Cálculo de la Reserva:\\ \-\hspace{0.5cm}
				Se realizara un resumen con los resultados del Cálculo en una plantilla de Excel, el cual incorporara el Cálculo de la primera Reserva, así como la conclusión de las $n$ simulaciones posteriores.
				\item Envío de Información a Áreas Técnicas:\\ \-\hspace{0.5cm}
				 Se enviaran los resultados de Reserva Técnica al resto de Áreas Técnicas para su registro en los reportes internos y su uso posterior en entregas de información al Organizmo Regulador\\
				
			\end{enumerate}
			\item \textit{Obtención de información para el Cálculo del Requerimiento de Capital de Solvencia (RCS)} 
				\begin{enumerate} 
					\item Obtención de Factóres de Devengamiento de la CUSF:\\ \-\hspace{0.5cm}
					Para el Cálculo de Insúmos de Suscripción es necesario contar con los factóres de Devengamiento publicados en la última actualización del \textit{Manual de datos para el cálculo del RCS de índices de siniestralidad del mejor estimador}.
					\item Obtención de Insúmos de Suscripción:\\ \-\hspace{0.5cm}
					Una vez obtenidos los Factóres de Devengamiento actualizados se calcúlan los Insumos de Suscripción mediante el uso de la Reserva Técnica calculada y los Asegurados en Vigor al Cierre de mes.
					\item Envío de los Insumos de Suscripción al Área Responsable del Cálculo del nuevo RCS:\\ \-\hspace{0.5cm} Una vez que se calcúlan los Insumos de Suscripción, estos deben enviarse al Area Responsable del Cálculo del RCS para que esta los ingrese al validador en linea de la CNSF y posteriormente envíe el nuevo RCS en el reporte correspondiente.\\ 
				\end{enumerate}
		\end{enumerate}
	\item \textbf{Actividades Trimestrales}
\begin{enumerate}
		\item \textit{Entrega de Reportes Trimestrales a la CNSF} \\ \-\hspace{0.5cm}
		La CNSF hace del conocimiento de las Instituciones de Seguros y Fianzas la responsabilidad de entregar reportes trimestrales con información que sustente la correcta operación de dichas Compañías.
		\begin{enumerate}
	\item {Ingresar a la pagina oficial de la CNSF para obtener las fechas límite de entrega de los diferentes reportes} \\ \-\hspace{0.5cm}
	La CNSF actualiza periodicamente el listado de reportes que se deben entregar, así como las fechas antes de las cuales debe hacerse el envío. Es responsabilidad del actuario estar informado de dichas actualizaciones para enviar los reportes en tiempo y forma.
	\item {Llenado y entrega del REPORTE REGULATORIO SOBRE RESERVAS TÉCNICAS Trimestral (RR-3 REVAL)} \\ \-\hspace{0.5cm}
	Siguiendo el Manual oficial de la CNSF debe ser llenado el reporte RR3 REVAL, el cuál esta consituido por la información del cálculo de Reservas Técnicas (Comparte información con el REPORTE REGULATORIO SOBRE ESTADOS FINANCIEROS RR7 por lo que se debe validar que no existan inconsistencias entre los reportes). Posteriormente, la información debe ser depositada en el Sistema de Entrega de Información Vía Electronica (SEIVE).
	\item {Llenado y entrega del Reporte de Información Estadística por Operación, Ramo o Seguro (RR8 ORS)} \\ \-\hspace{0.5cm}
	Siguiendo el Manual oficial de la CNSF debe ser llenado el reporte RR8 ORS, el cuál esta consituido por la información estadística de los Asegurados en la cartera de Compañía. Posteriormente, la información debe ser depositada en el Sistema de Entrega de Información Vía Electronica (SEIVE).
		\item {Llenado y entrega del Reporte de Información del Comportamiento por Operación y Ramo (RR8 COR)} \\ \-\hspace{0.5cm}
	Siguiendo el Manual oficial de la CNSF debe ser llenado el reporte RR8 COR, el cuál esta consituido por el número de Pólizas y Asegurados y Monto de Prima por Subramo en la Cartera de la Compañía. Posteriormente, la información debe ser depositada en el Sistema de Entrega de Información Vía Electronica (SEIVE).
		\end{enumerate}
	\item \textit{Entrega de Reportes Trimestrales a la AMIS} \\ \-\hspace{0.5cm}
	Con el fin de realizar análisis estadisticos del mercado en beneficio de las instituciones de Seguros y Fianzas, la AMIS solicita a las Compañías que le envíen los reportes Trimestrales RR mediante un Repositorio en Linea. Es responsabilidad del Actuario hacer el envío Trimestral de la información una vez se hayan entregado de manera exitosa los Reportes a la CNSF.
	\item \textit{Entrega de información al Auditor Externo} \\ \-\hspace{0.5cm}
	Con el propósito de dar cumplimiento a lo establecido por la CNSF, las Compañías estan obligadas a contratar un Auditor Externo cuya función será buscar inconsistencias en su información interna para su pronta y oportuna corrección. Una vez corregida la información el Auditor Externo ratificará ante la CNSF que esta es confiable. Es responsabilidad del Actuario compartir con el Auditor Externo la información convenida en el Programa de Auditoría para posteriormente, en conjunto, resolver dudas y acordar soluciones. 
\end{enumerate}
\item \textbf{Actividades Anuales}
\begin{enumerate}
	\item \textit{Asistencia a los foros organizados por la AMIS para revisar nuevos reportes o modificaciones a reportes anuales que solicita la CNSF, aclarar dudas y acordar propuestas para presentar a la CNSF.} \\ \-\hspace{0.5cm}
		\item \textit{Entrega de Reportes Trimestrales a la CNSF} \\ \-\hspace{0.5cm}
	La CNSF hace del conocimiento de las Instituciones de Seguros y Fianzas la responsabilidad de entregar reportes anuales con información que sustente la correcta operación de dichas Compañías.
	\begin{enumerate}
		\item {Ingresar a la pagina oficial de la CNSF para obtener las fechas límite de entrega de los diferentes reportes} \\ \-\hspace{0.5cm}
		La CNSF actualiza periodicamente el listado de reportes que se deben entregar, así como las fechas antes de las cuales debe hacerse el envío. Es responsabilidad del actuario estar informado de dichas actualizaciones para enviar los reportes en tiempo y forma.
		\item {Llenado y entrega del REPORTE REGULATORIO SOBRE RESERVAS TÉCNICAS Anual (RR-3 DETAN)} \\ \-\hspace{0.5cm}
		Siguiendo el Manual oficial de la CNSF debe ser llenado el reporte RR3 DETAN, el cuál esta consituido por la información del cálculo de Reservas Técnicas de cada asegurado en la cartera de la Compañía (Comparte información con el REPORTE REGULATORIO SOBRE ESTADOS FINANCIEROS RR7 por lo que se debe validar que no existan inconsistencias entre los reportes). Posteriormente, la información debe ser depositada en el Sistema de Entrega de Información Vía Electronica (SEIVE).
		\item {Llenado y entrega del Reporte SISTEMA ESTADÍSTICO DEL SECTOR ASEGURADOR (RR8 SESA)} \\ \-\hspace{0.5cm}
		Siguiendo el Manual oficial de la CNSF debe ser llenado el reporte RR8 SESA para cada ramo que la Compañía maneje en su cartera, el cuál esta consituido por la información estadística de cada Asegurado en la cartera de la Compañía. Posteriormente, la información debe ser depositada en el Sistema de Entrega de Información Vía Electronica (SEIVE).
		\item {Llenado y entrega del Reporte FORMAS ESTADÍSTICAS DE LOS SEGUROS (RR8 FES)} \\ \-\hspace{0.5cm}
		Siguiendo el Manual oficial de la CNSF debe ser llenado el reporte RR8 FES para cada ramo que la Compañía maneje en su cartera, el cuál esta consituido por la información estadística por estado de cada Asegurado en la cartera de la Compañía. Posteriormente, la información debe ser depositada en el Sistema de Entrega de Información Vía Electronica (SEIVE).
		\end{enumerate}
	\item \textit{Entrega regulatoria de la carta de aceptación del Actuario Independiente firmada en conjunto del Contrato de Servicios del Actuario Independiente} \\ \-\hspace{0.5cm}
	Como parte de la regulación, es necesario hacer el envío a la CNSF de la carta de aceptación del Actuario que fungira como Auditor Externo de la Compañía con las certificaciones correspondientes. De igual forma debe incluirse el contrato firmado en el que se define el periodo, terminos y condiciones para que el Actuario Independiente cumpla con sus funciones como Auditor Externo. La información debe ser depositada en el Sistema de Entrega de Información Vía Electronica (SEIVE).
	\item \textit{Entrega regulatoria del Programa de Actividades del Actuario Independiente} \\ \-\hspace{0.5cm}
	Como parte de la regulación, es necesario hacer el envío a la CNSF del Programa de Actividades del Actuario Independiente, el cual debe incluir la información que se auditará, así como los periodos de entrega de información por parte de la Compañía y las fechas de entrega de comentarios por parte del Auditor. 
\end{enumerate}
	\item \textbf{Otras actividades}
\begin{enumerate}
	\item \textit{Automatización y Documentación de Procesos} \\ \-\hspace{0.5cm}
	Debido a la cantidad y frecuencia de reportes regulatorios, es responsabilidad del Actuario buscar nuevas y mejores formas de procesar y extraer la información que se requiere en la operación diaria, así como documentar dichos procesos para facilitar su uso.
	\item \textit{Creación de Manuales de Capacitación} \\ \-\hspace{0.5cm}
El Actuario debe constantemente crear y/o actualizar manuales que detallen sus funciones para la facil asimilacion de dichas funciones por nuevos miembros de equipo, al igual que los Auditores Internos de la Compañía para así asegurar la correcta operación y la calidad de la información.
\end{enumerate}
\end{enumerate}
\end{singlespace}



	\chapter{Conceptos Básicos}

	\section{El Seguro}
	El seguro es un medio para la protección individuos frente a las consecuencias de riesgos y se basa en transferir dichos riesgos a la institución de seguros, la cual se encargará de indemnizar todo o parte del perjuicio que se produzca por la ocurrencia de un evento previsto\footnote{Ver \citet{ASeguro}, Fundación Mapfre, El Seguro}.
	
	\section{Instituciones de Seguros}

	Una Institución de Seguros es una sociedad anónima autorizada para organizarse y operar conforme a la Ley de Instituciones de Seguros y de Fianzas (LISF), como institución de seguros\footnote{Ver \citet{BAseguradora}, Artículo 2 Sección XVI}.

 
	
	\section{Contrato de Seguro}
	
	El Contrato de Seguro es aquel con que la Empresa Aseguradora se obliga, mediante el pago de una prima, a resarcir un daño o a pagar una suma de dinero al verificarse la eventualidad prevista en el contrato\footnote{Ver \citet{CContrato}, Artículo 1}.
	
	\section{Reserva Técnica}

	Debe ser constituida por la Institución de Seguros para operar de acuerdo a la LISF. Las Reservas Técnicas detalladas en este trabajo son la Reserva de Riesgos en Curso (RRC) y la Reserva para Obligaciones Pendientes de Cumplir (ROPC). El propósito de la RRC es cubrir el valor esperado de las obligaciones futuras derivadas del pago de siniestros, beneficios, valores garantizados, dividendos, gastos de adquisición y administración, así como cualquier otra obligación futura derivada de los contratos del seguro. Por otro lado, la ROPC tiene como objetivo cubrir el valor esperado de siniestros, beneficios, valores garantizados o dividendos, una vez ocurrida la eventualidad prevista en el contrato de seguro. De acuerdo a la LISF, las reservas técnicas deberán constituirse y valuarse de forma prudente, confiable y objetiva, en relación con todas las obligaciones de seguro que las Instituciones de Seguros asuman frente a los asegurados y beneficiarios del contrato de seguro, los gastos de administración, si como los gastos de adquisición que, en su caso, asuman con relación a los mismos. Para la constitución se deben utilizar métodos actuariales con base en la aplicación de los estándares de práctica actuarial, considerando la información disponible en los mercados financieros, así como la que generalmente se encuentra disponible sobre riesgos técnicos de seguros\footnote{Ver \citet{DReservasTec}, Artículo 217 y 218}. 
	
	\section{Comisión Nacional de Seguros y Fianzas}

	La Comisión Nacional de Seguros y Fianzas es un Órgano Desconcentrado de la Secretaría de Hacienda y Crédito Público, encargada de supervisar que la operación de los sectores asegurador y afianzador se apegue al marco normativo, preservando la solvencia y estabilidad financiera de las instituciones de Seguros y Fianzas, para garantizar los intereses del público usuario, así como promover el sano desarrollo de estos sectores con el propósito de extender la cobertura de sus servicios a la mayor parte posible de la población\footnote{Ver \citet{EComision}}. 
	
	\section{Secretaría de Hacienda y Credito Publico}
	
	La Secretaría de Hacienda y Crédito Público es la dependencia del Poder Ejecutivo Federal que tiene como misión proponer, dirigir y controlar la política económica del Gobierno Federal en materia financiera, fiscal, de gasto, de ingresos y deuda pública, con el propósito de consolidar un país con crecimiento económico de calidad\footnote{Ver \citet{NSHCP}}.
	
	
	\section{Asociación Mexicana de Instituciones de Seguros}
	
	Organismo gremial que representa al interés general de las compañías aseguradoras, promoviendo el desarrollo sano y sustentable del seguro a través de las mejores prácticas. Su principal objetivo es promover el desarrollo de la industria aseguradora, representar sus  intereses ante autoridades del sector público, privado y social, así como proporcionar apoyo técnico a sus asociados\footnote{Ver \citet{FAmis}}. 
	
	\chapter{Marco Regulatorio y Herramientas utilizadas}\label{tcyedb}
	
	\section{Circular Única de Seguros y Fianzas}

	
	Cuerpo normativo que contiene las disposiciones derivadas de la Ley de Instituciones de Seguros y de Fianzas, que dan operatividad a sus preceptos y sistematizan su integración, homologando la terminología utilizada, a fin de brindar con ello certeza jurídica en cuanto al marco normativo al que las instituciones y sociedades mutualistas de seguros, instituciones de fianzas y demás personas y entidades sujetas a la inspección y vigilancia de la Comisión Nacional de Seguros y Fianzas deberán sujetarse en el desarrollo de sus operaciones\footnote{Ver \citet{IDefCusf}} .El 19 de diciembre de 2014, se publicó en el Diario Oficial de la Federación la Circular Única de Seguros y Fianzas (CUSF). Esta circular instrumenta y da operatividad a la nueva Ley de Instituciones de Seguros y de Fianzas (LISF) promulgada el 4 de abril de 2013 y en vigor desde el 4 de abril de 2015. 
	
	\section{Ley de Instituciones de Seguros y de Fianzas}

	Ley que tiene por objeto la organización, operación y funcionamiento de las Instituciones de Seguros, Instituciones de Fianzas y Sociedades Mutualistas de Seguros; las actividades y operaciones que las mismas podrán realizar, así como las de los agentes de seguros y de fianzas, y demás participantes en las actividades aseguradora y afianzadora en protección de los intereses del público usuario de estos servicios financieros\footnote{Ver \citet{GLisf}, Artículo 1}. De acuerdo con La Ley de Instituciones de Seguros y de Fianzas se establece que las Instituciones de Seguros deberán registrar ante la Comisión Nacional de Seguros y Fianzas los métodos actuariales con base en sus estimaciones para la Reserva de Riesgos en Curso\footnote{Ver \citet{FAmis}, Artículo 219}, de conformidad con las disposiciones de carácter general que al efecto emita, mismas que se dieron a conocer a través de la Circular Única de Seguros y Fianzas publicada en el Diario Oficial de la Federación\footnote{Ver \citet{HCusf}}.
	
	
	\section{R}

	R es una herramienta informática (específicamente, un lenguaje computacional) sumamente potente para realizar distintos cálculos científicos, numéricos y estadísticos, así como para crear gráficas y figuras de gran calidad. R es un programa gratuito, relativamente fácil de operar y cuenta con una gran comunidad de internet que contribuye a resolver dudas y problemas, sin costo alguno\footnote{Ver \citet{KR}}. 
	

	\chapter{Metodología empleada en el Cálculo de la Reserva de Riesgos en Curso}\label{tcyedb}
	
	\section{Método Chain Ladder}

	Método que se utiliza comunmente en las reservas de no-vida. Útiliza un factor para "suavizar" los datos y con base en estos, realizar interpolaciones para estimar los siniestros agregados para cada año de ocurrencia y posteriormente la reserva correspondiente. El supuesto básico de este método es que las columnas en el triángulo de desarrollo son proporcionales, es decir que, independientemente del año de origen, cada periodo de desarrollo se reporta una proporción constante de siniestros con respecto al total. La sustentación del supuesto depende en buena medida, tanto del tipo de negocio que se trate, como de la homogeneidad y tamaño de la cartera. En particular, en negocios como vida individual, gastos médicos, responsabilidad civil, etc., la evolución del reporte de los siniestros es estacional\footnote{Ver \citet{LChainLadder}, 1.2.1 Método Chain-Ladder}.  

	La estimación de las obligaciones se hace con base en los siniestros observados y su desfase respecto a la entrada en vigor de cada obligación, usando el metodo bootstrap.
	
	\section{Bootstrapping}
	El método desarrollado por Bradley Efron. Es un método de muestreo computacionalmente intensivo con el que se busca aproximar la distribución muestral de alguna variable aleatoria que se basa en los datos observados\footnote{Ver \citet{MBootstrap}, p.11}.

	El método de Bootstrap es un método de muestreo con el que se busca aproximar la distribución muestral de alguna variable aleatoria que tiene como base los datos observados.

	Teniendo una muestra de datos $x_{1},x_{2},x_{3},...,x_{n}$, donde los $x_{i}$ son independientes y provienen de una distribución desconocida $F$, donde además se presume que dicha muestra es una representación significativa de la población de donde proviene. Se tiene además una variable aleatoria $R(X,F)$ que depende de $X$ y de la función desconocida $F$. Entonces se puede realizar una muestra aleatoria de tamaño $n$ con reemplazo de la muestra de datos, $x_{1}^{*},x_{2}^{*},x_{3}^{*},...,x_{n}^{*}$ y a partir de esa muestra se puede calcular una observación de la variable aleatoria $R*(X*,P*)$, donde $F*$ es la distribución de probabilidad de la muestra, que se construyó de tipo uniforme. Finalmente, se realizan más muestras y se calculan más valores de $R*$ para poder estimar la distribución $R(X,F)$.

	La utilidad técnica de bootstrapping es que permite aproximar la distribución de alguna estadística de los datos de una forma fácil y rápida. Adicionalmente, no es necesario hacer una estimación paramétrica ni supuestos acerca de la distribución de los datos.

	\chapter{Cálculo de la Reserva de Riesgos en Curso}\label{metnum}
	
	La valuación y constitución de la reserva de riesgos en curso deberá calcularse para un grupo homogéneo definido, correspondiente a un cierto subramo y tipo de seguro que la Compañia en cuestión tenga en su cartera. El proceso aquí descrito fué aplicado en la practica para grupos homogéneos correspondientes a los ramos de Gastos Médicos Colectivo y Salud Colectivo.
	
	Para realizar los calculos es necesario identificar primero el número de asegurados en vigor al cierre del mes al momento de la valuación, el monto de prima correspondiente a los beneficios contratados, los gastos asociados y el periodo de cobertura de cada asegurado en vigor.
		
	\section{Cálculo del Bel de Riesgos en Curso}

	El cálculo del Bel de Riesgo implica un análisis de las obligaciones futuras para los riesgos en curso con base en los siniestros que actualmente han sido reportados. Para ello se necesita la construcción de una matriz de desarrollo de siniestros  de dimensiones $(k$ x $s)$, en la cual los siniestros se distribuyen por el trimestre en que se reporto cada uno de los procedimientos ocurridos respecto al inicio de vigencia de la póliza y consideramos montos netos de siniestralidad, es decir, no tomamos en cuenta el monto de deducible y copago a cargo del asegurado. La matriz queda de la siguiente manera:

	\vspace{0.1cm}
	
	\begin{itemize}
		\setlength\itemsep{-0.5em}
		\item ${X}_{i,j}$ es el monto de siniestros de las pólizas con inicio de vigencia en el trimestre $i$ que fue reportado $j$ trimestres posteriores al inicio de vigencia.
		\item ${k}_{}$ es el número de trimestres máximo observado en la experiencia de siniestros.
		\item ${s}_{}$ es el número de trimestres de experiencia de inicio de vigencia.
		\item $i$ es el trimestre de inicio de vigencia de la póliza, $i\in \left\{1,2,3,\dots ,12\right\}$
		\item $j$ es el trimestre en que se reportó el siniestro,  $j\in \left\{1,2,3,\dots\right\}$
	\end{itemize}
	
	\vspace{1cm}

	\begin{table}[ht]
	\centering
		\caption{Matriz de Siniestros}
	\begin{tabularx}{\linewidth}{c|ccccccccc}
		\multirow{2}{4cm}{Trimestre de inicio de vigencia de la póliza} & \multicolumn{9}{c}{Trimestre en que se reportó el procedimiento} \\
			& 0  & 1 & 2 & $ \dots $ & j & $\dots $ & k-2 & k-1 &  k \\
		\midrule
		1      &  $X_{1,0}^{}$ & $X_{1,1}^{}$ & $X_{1,2}^{}$ & $ \dots $ & $X_{1,j}^{}$ & $ \dots $ & $X_{1,k-2}^{}$ & $X_{1,k-1}^{}$ & $X_{1,k}^{}$ \\
		2      &  $X_{2,0}^{}$ & $X_{2,1}^{}$ & $X_{2,2}^{}$ & $ \dots $ & $X_{2,j}^{}$ & $ \dots $ & $X_{2,k-2}^{}$ & $X_{2,k-1}^{}$ & \\
		3      &  $X_{3,0}^{}$ & $X_{3,1}^{}$ & $X_{3,2}^{}$ & $ \dots $ & $X_{3,j}^{}$ & $ \dots $ & $X_{3,k-2}^{}$ & & \\
		4      &  $X_{4,0}^{}$ & $X_{4,1}^{}$ & $X_{4,2}^{}$ & $ \dots $ & $X_{4,j}^{}$ & $ \dots $ & & & \\
		:      & & & & & & & & &\\
		i      &  $X_{i,0}^{}$ & $X_{i,1}^{}$ & $X_{i,2}^{}$ & $ \dots $ & $X_{i,j}^{}$ & & & & \\
		:      & & & & & & & & & \\
		s-2      &  $X_{s-2,0}^{}$ & $X_{s-2,1}^{}$ & $X_{s-2,2}^{}$ & & & & & & \\
		s-1      &  $X_{s-1,0}^{}$ & $X_{s-1,1}^{}$ & & & & & & & \\
		s      &  $X_{s,0}^{}$ & & & & & & & & \\
	\end{tabularx}
	\end{table}

	\vspace{1cm}

	Una vez que se obtiene la matriz de siniestros, la usamos para generar una matriz de siniestros acumulados en la cual definimos ${Y}_{i,j}$ como el monto de siniestros de la póliza con inicio de vigencia en el trimestre $i$ reportados hasta el trimestre $j$:
	
	\begin{equation}
	{y}_{i,j}=\sum _{m=0}^{j}{X}_{i,m}
		\label{eq1}
	\end{equation}

	Donde
	
	\begin{itemize}
	\setlength\itemsep{-0.5em}
		\item ${X}_{i,m}$ es el monto de siniestros de las pólizas con inicio de vigencia en el trimestre $i$ que fue reportado $m$ trimestres posteriores al inicio de vigencia.
		\item $i$ es el trimestre de inicio de vigencia de la póliza, $i\in \left\{1,2,3,\dots ,12\right\}$
		\item $j$ es el trimestre en que se reportó el siniestro,  $j\in \left\{1,2,3,\dots\right\}$
		\item $m$ es el trimestre de acumulación, $m\in \left\{0,1,2,\dots ,j\right\}$, $m\le j\le k$
	\end{itemize} 
	
	\begin{table}[ht]
	\centering
		\caption{Matriz de Siniestros Acumulados}
	\begin{tabularx}{\linewidth}{c|cccccccccc}
		\multirow{2}{4cm}{Trimestre de inicio de vigencia de la póliza}&\multicolumn{9}{c}{ Trimestre en que se reportó el procedimiento} \\
		& 0  & 1 & 2 & $ \dots $ & j & $\dots $ & k-2 & k-1 &  k \\
		\midrule
		1      &  $Y_{1,0}^{}$ & $Y_{1,1}^{}$ & $Y_{1,2}^{}$ & $ \dots $ & $Y_{1,j}^{}$ & $ \dots $ & $Y_{1,k-2}^{}$ & $Y_{1,k-1}^{}$ & $Y_{1,k}^{}$ \\
		2      &  $Y_{2,0}^{}$ & $Y_{2,1}^{}$ & $Y_{2,2}^{}$ & $ \dots $ & $Y_{2,j}^{}$ & $ \dots $ & $Y_{2,k-2}^{}$ & $Y_{2,k-1}^{}$ & \\
		3      &  $Y_{3,0}^{}$ & $Y_{3,1}^{}$ & $Y_{3,2}^{}$ & $ \dots $ & $Y_{3,j}^{}$ & $ \dots $ & $Y_{3,k-2}^{}$ & & \\
		4      &  $Y_{4,0}^{}$ & $Y_{4,1}^{}$ & $Y_{4,2}^{}$ & $ \dots $ & $Y_{4,j}^{}$ & $ \dots $ & & & \\
		:      & & & & & & & & &\\
		i      &  $Y_{i,0}^{}$ & $Y_{i,1}^{}$ & $Y_{i,2}^{}$ & $ \dots $ & $Y_{i,j}^{}$ & & & & \\
		:      & & & & & & & & & \\
		s-2      &  $Y_{s-2,0}^{}$ & $Y_{s-2,1}^{}$ & $Y_{s-2,2}^{}$ & & & & & & \\
		s-1      &  $Y_{s-1,0}^{}$ & $Y_{s-1,1}^{}$ & & & & & & & \\
		s      &  $Y_{s,0}^{}$ & & & & & & & & \\
	\end{tabularx}
	\end{table}
	
	Usando la matriz de siniestros acumulados obtenemos los factores de incremento ${f}_{j}$, los cuales indican el incremento dado de un trimestre a otro:

	\begin{equation}
	{f}_{j}=\frac{\sum _{i=1}^{s-j}{Y}_{i,j}}{\sum _{i=1}^{s-j}{Y}_{i,j-1}}, \hspace{0.5cm}\text{con}\hspace{0.5cm} 0< j\le k
	\label{eq2}
	\end{equation}

	Donde ${Y}_{i,j}$ representa el monto de siniestros de las pólizas con inicio de vigencia en el trimestre $i$ reportados hasta el trimestre $j$, como se ve en \eqref{eq2}

	Mediante estos factores de incremento, se definen los Siniestros Esperados para la vigencia $i$ (${SE}_{i}$) como la estimación del monto de siniestros que serán reportados para las pólizas con inicio de vigencia en el trimestre $i$:
	
	\begin{equation}
	{SE}_{i}={Y}_{s-i+1,i-1}\cdot\Pi_{j=i}^{k}{f}_{j},\hspace{0.5cm}\text{con}\hspace{0.5cm}0< i\le k
		\label{eq3}
	\end{equation}

	Donde ${Y}_{s-i+1,i-1}$ representa el monto de siniestros de las pólizas con inicio de vigencia en el trimestre $s-i+1$ reportados hasta el trimestre $i-1$, y ${f}_{i}$ es el factor de incremento del trimestre $j$. 

	Definimos las obligaciones futuras iniciales de riesgos en curso (${RRC}^{0}$) como la diferencia entre los Siniestros Estimados y los Siniestros Acumulados observados:
	
	\begin{equation}
	{RRC}^{0}=\sum _{i=s-k}^{s}{SE}_{i}-{Y}_{i,s-i}
	\label{eq4}
	\end{equation}

	Donde ${SE}_{i}$ son los siniestros esperados para la vigencia $i$, y ${Y}_{i,s-i}$ es el monto de siniestros de las pólizas con inicio de vigencia en el trimestre $i$ reportados hasta el trimestre $s-i$
		 
	Definimos ${Y}_{i,j}^{*}$ como el monto ajustado de siniestros de las pólizas con inicio de vigencia en el trimestre $i$ reportados hasta el trimestre $j$:
	
		\begin{equation}
	{Y}_{i,j-1}^{*}=\frac{{Y}_{i,j}^{*}}{{f}_{j}}
	\label{eq5}
	\end{equation}

	Con:

	$${Y}_{i,k-i+1}^{*}={Y}_{i,k-i+1}$$	

	Donde ${Y}_{i,j}$ es el monto de siniestros de las pólizas con inicio de vigencia en el trimestre $i$ reportados hasta el trimestre $j$, y ${f}_{j}$ es el factor de incremento del trimestre $j$.

	Con estos montos obtenemos la matriz de siniestros acumulados ajustados mostrada en la tabla \ref{matrizsiniestrosacumulados}, partiendo del último dato observado.
		
	\begin{table}[ht]
		\centering
		\caption{Matriz de Siniestros Acumulados Ajustados}
		\begin{tabularx}{\linewidth}{ c |ccccccccc}
			\multirow{2}{4cm}{Trimestre de inicio de vigencia de la póliza}
			& \multicolumn{9}{c}{Trimestre en que se reportó el procedimiento} \\ 
			& 0  & 1 & 2 & $ \dots $ & j & $\dots $ & k-2 & k-1 &  k \\
			\midrule
			1      &  $Y_{1,0}^{*}$ & $Y_{1,1}^{*}$ & $Y_{1,2}^{*}$ & $ \dots $ & $Y_{1,j}^{*}$ & $ \dots $ & $Y_{1,k-2}^{*}$ & $Y_{1,k-1}^{*}$ & $Y_{1,k}^{}$ \\
			2      &  $Y_{2,0}^{*}$ & $Y_{2,1}^{*}$ & $Y_{2,2}^{*}$ & $ \dots $ & $Y_{2,j}^{*}$ & $ \dots $ & $Y_{2,k-2}^{*}$ & $Y_{2,k-1}^{}$ & \\
			3      &  $Y_{3,0}^{*}$ & $Y_{3,1}^{*}$ & $Y_{3,2}^{*}$ & $ \dots $ & $Y_{3,j}^{*}$ & $ \dots $ & $Y_{3,k-2}^{}$ & & \\
			4      &  $Y_{4,0}^{*}$ & $Y_{4,1}^{*}$ & $Y_{4,2}^{*}$ & $ \dots $ & $Y_{4,j}^{*}$ & $ \dots $ & & & \\
			:      & & & & & & & & & \\
			i      &  $Y_{i,0}^{*}$ & $Y_{i,1}^{*}$ & $Y_{i,2}^{*}$ & $ \dots $ & $Y_{i,j}^{}$ & & & & \\
			:      & & & & & & & & & \\
			s-2      &  $Y_{s-2,0}^{*}$ & $Y_{s-2,1}^{*}$ & $Y_{s-2,2}^{}$ & & & & & & \\
			s-1      &  $Y_{s-1,0}^{*}$ & $Y_{s-1,1}^{}$ & & & & & & & \\
			s      &  $Y_{s,0}^{}$ & & & & & & & & \\
		\end{tabularx}
	\end{table}\label{matrizsiniestrosacumulados}
	
	Y a partir de esta matriz de siniestros acumulados ajustados se obtiene una nueva matriz de montos $X_{i,j}^{*}$, que son los siniestros ajustados de las pólizas con inicio de vigencia en el trimestre $i$ que fueron reportados $j$ trimestres posteriores al inicio de vigencia:
	
	\begin{equation}
	{X}_{i,j}^{*}={Y}_{i,j}^{*}-\sum _{m=0}^{j-1}{X}_{i,m}^{*}
	\label{eq6}
	\end{equation}	
	 
	Con:
	
	$${X}_{i,0}^{*}={Y}_{i,0}^{*}$$ 
	
	Donde ${Y}_{i,j}^{*}$ es el monto ajustado de siniestros de las pólizas con inicio de vigencia en el trimestre $i$ reportados hasta el trimestre $j$. 
	
	\begin{table}[ht]
		\centering
			\caption{Matriz de Siniestros  Ajustados}
		\begin{tabularx}{\linewidth}{ c|ccccccccc}
		\multirow{2}{4cm}{Trimestre de inicio de vigencia de la póliza}&  \multicolumn{9}{c}{Trimestre en que se reportó el procedimiento} \\
			& 0  & 1 & 2 & $ \dots $ & j & $\dots $ & k-2 & k-1 &  k \\
			\midrule
			1      &  $X_{1,0}^{*}$ & $X_{1,1}^{*}$ & $X_{1,2}^{*}$ & $ \dots $ & $X_{1,j}^{*}$ & $ \dots $ & $X_{1,k-2}^{*}$ & $X_{1,k-1}^{*}$ & $X_{1,k}^{*}$ \\
			2      &  $X_{2,0}^{*}$ & $X_{2,1}^{*}$ & $X_{2,2}^{*}$ & $ \dots $ & $X_{2,j}^{*}$ & $ \dots $ & $X_{2,k-2}^{*}$ & $X_{2,k-1}^{*}$ & \\
			3      &  $X_{3,0}^{*}$ & $X_{3,1}^{*}$ & $X_{3,2}^{*}$ & $ \dots $ & $X_{3,j}^{*}$ & $ \dots $ & $X_{3,k-2}^{*}$ & & \\
			4      &  $X_{4,0}^{*}$ & $X_{4,1}^{*}$ & $X_{4,2}^{*}$ & $ \dots $ & $X_{4,j}^{*}$ & $ \dots $ & & & \\
			:      & & & & & & & & &\\
			i      &  $X_{i,0}^{*}$ & $X_{i,1}^{*}$ & $X_{i,2}^{*}$ & $ \dots $ & $X_{i,j}^{*}$ & & & & \\
			:      & & & & & & & & & \\
			s-2      &  $X_{s-2,0}^{*}$ & $X_{s-2,1}^{*}$ & $X_{s-2,2}^{*}$ & & & & & & \\
			s-1      &  $X_{s-1,0}^{*}$ & $X_{s-1,1}^{*}$ & & & & & & & \\
			s      &  $X_{s,0}^{*}$ & & & & & & & & \\
		\end{tabularx}
	\end{table}
		
	La diferencia del monto de siniestros observados (el monto original) y el monto de siniestros ajustado son los residuales brutos $R_{i,j}^{}$:
		
	\begin{equation}
	R_{i,j}^{}= X_{i,j}^{} - X_{i,j}^{*} 
	\label{eq7}
	\end{equation}
	
	Donde ${X}_{i,j}^{*}$ es el monto de siniestros ajustados de las pólizas con inicio de vigencia en el trimestre $i$ que fueron reportados $j$ trimestres posteriores al inicio de vigencia, y ${X}_{i,j}$ es el monto de siniestros de las pólizas con inicio de vigencia en el trimestre $i$ que fueron reportados $j$ trimestres posteriores al inicio de vigencia.	
	
	Generamos la matriz de Residuales de la siguiente manera:
	

	\begin{table}[ht]
		\centering
			\caption{Matriz de Residuales}
		\begin{tabularx}{\linewidth}{ c|ccccccccc}
			\multirow{2}{4cm}{Trimestre de inicio de vigencia de la póliza}
			& \multicolumn{9}{c}{Trimestre en que se reportó el procedimiento} \\
			& 0  & 1 & 2 & $ \dots $ & j & $\dots $ & k-2 & k-1 &  k \\
			\midrule
			1      &  $R_{1,0}^{ }$ & $R_{1,1}^{ }$ & $R_{1,2}^{ }$ & $ \dots $ & $R_{1,j}^{ }$ & $ \dots $ & $R_{1,k-2}^{ }$ & $R_{1,k-1}^{ }$ & $R_{1,k}^{ }$ \\
			2      &  $R_{2,0}^{ }$ & $R_{2,1}^{ }$ & $R_{2,2}^{ }$ & $ \dots $ & $R_{2,j}^{ }$ & $ \dots $ & $R_{2,k-2}^{ }$ & $R_{2,k-1}^{ }$ & \\
			3      &  $R_{3,0}^{ }$ & $R_{3,1}^{ }$ & $R_{3,2}^{ }$ & $ \dots $ & $R_{3,j}^{ }$ & $ \dots $ & $R_{3,k-2}^{ }$ & & \\
			4      &  $R_{4,0}^{ }$ & $R_{4,1}^{ }$ & $R_{4,2}^{ }$ & $ \dots $ & $R_{4,j}^{ }$ & $ \dots $ & & & \\
			:      & & & & & & & & & \\
			i      &  $R_{i,0}^{ }$ & $R_{i,1}^{ }$ & $R_{i,2}^{ }$ & $ \dots $ & $R_{i,j}^{ }$ & & & & \\
			:      & & & & & & & & & \\
			s-2      &  $R_{s-2,0}^{ }$ & $R_{s-2,1}^{ }$ & $R_{s-2,2}^{ }$ & & & & & & \\
			s-1      &  $R_{s-1,0}^{ }$ & $R_{s-1,1}^{ }$ & & & & & & & \\
			s      &  $R_{s,0}^{ }$ & & & & & & & & \\
		\end{tabularx}
	\end{table}
 

 
	Utilizamos el método de bootstrap, bajo el supuesto de que los residuales brutos $R_{i,j}$ de la matriz provienen de la misma distribución y son independientes. Obtenemos el valor mínimo y máximo observado de cada columna $j$ como el intervalo de residuales observado.
	 
	Entonces definimos a $R_{j}^{\text{min}}$ como el valor mínimo de los residuales observados en el trimestre reportado $j$ y $R_{j}^{\text{max}}$ como el valor máximo de los residuales observados en el trimestre reportado $j$:
	
	\begin{table}[ht]
		\centering
		
		\begin{tabularx}{\linewidth}{ c|ccccccccc}
			& \multicolumn{9}{c}{Trimestre en que se reportó el procedimiento} \\
			& 0  & 1 & 2 & $ \dots $ & j & $\dots $ & k-2 & k-1 &  k\\
			\midrule
			Mínimo      &  $R_{0}^{\text{min}}$ & $R_{1}^{\text{min}}$ & $R_{2}^{\text{min}}$ & $ \dots $ & $R_{j}^{\text{min}}$ & $ \dots $ & $R_{k-2}^{\text{min}}$ & $R_{k-1}^{\text{min}}$ & $R_{k}^{\text{min}}$ \\
			Máximo      &  $R_{0}^{\text{max}}$ & $R_{1}^{\text{max}}$ & $R_{2}^{\text{max}}$ & $ \dots $ & $R_{j}^{\text{max}}$ & $ \dots $ & $R_{k-2}^{\text{max}}$ & $R_{k-1}^{\text{max}}$ & $R_{k}^{\text{max}}$ \\
		\end{tabularx}
	\end{table}
	
	Con

	$$R_{j}^{\text{min}}= \text{min}_{ i\in \left\{1,2,\dots ,S\right\}} \left[R_{i,j}^{}\right]$$	
	
	$$R_{j}^{\text{max}}= \text{max}_{ i\in \left\{1,2,\dots ,S\right\}} \left[R_{i,j}^{}\right]$$	 

	Donde $R_{i,j}$ es el residual bruto del trimestre de inicio de vigencia $i$ reportado en el trimestre $j$.

	Realizamos un muestreo con reemplazo de residuales tomando $n$ muestras, de forma uniforme dentro del intervalo $\left[R_{j}^{\text{min}},R_{j}^{\text{max}}\right]$ de cada una de las columnas de reportado $j$.
	
	Definimos $R_{i,j}^{*}$ como el residual de la muestra de pólizas con inicio de vigencia en el trimestre $i$ con trimestre de reporte $j$. 
	
	Generamos la matriz de residuales de la siguiente forma:
	
	\begin{table}[H]
		\centering
			\caption{Matriz de Siniestros Simulados}
		\begin{tabularx}{\linewidth}{ c |ccccccccc}
			\multirow{2}{4cm}{Trimestre de inicio de vigencia de la póliza}& \multicolumn{9}{c}{Trimestre en que se reportó el procedimiento} \\
			& 0  & 1 & 2 & $ \dots $ & j & $\dots $ & k-2 & k-1 &  k\\
			\midrule
			1      &  $R_{1,0}^{*}$ & $R_{1,1}^{*}$ & $R_{1,2}^{*}$ & $ \dots $ & $R_{1,j}^{*}$ & $ \dots $ & $R_{1,k-2}^{*}$ & $R_{1,k-1}^{*}$ & $R_{1,k}^{*}$ \\
			2      &  $R_{2,0}^{*}$ & $R_{2,1}^{*}$ & $R_{2,2}^{*}$ & $ \dots $ & $R_{2,j}^{*}$ & $ \dots $ & $R_{2,k-2}^{*}$ & $R_{2,k-1}^{*}$ & \\
			3      &  $R_{3,0}^{*}$ & $R_{3,1}^{*}$ & $R_{3,2}^{*}$ & $ \dots $ & $R_{3,j}^{*}$ & $ \dots $ & $R_{3,k-2}^{*}$ & & \\
			4      &  $R_{4,0}^{*}$ & $R_{4,1}^{*}$ & $R_{4,2}^{*}$ & $ \dots $ & $R_{4,j}^{*}$ & $ \dots $ & & & \\
			:      & & & & & & & & & \\
			i      &  $R_{i,0}^{*}$ & $R_{i,1}^{*}$ & $R_{i,2}^{*}$ & $ \dots $ & $R_{i,j}^{*}$ & & & & \\
			:      & & & & & & & & & \\
			s-2      &  $R_{s-2,0}^{*}$ & $R_{s-2,1}^{*}$ & $R_{s-2,2}^{*}$ & & & & & & \\
			s-1      &  $R_{s-1,0}^{*}$ & $R_{s-1,1}^{*}$ & & & & & & & \\
			s      &  $R_{s,0}^{*}$ & & & & & & & & \\
		\end{tabularx}
	\end{table}

	Obtenemos así una matriz de siniestros simulada al agregar el residual obtenido a cada monto de siniestros ajustados.
	
	Para esto definimos $X_{i,j}^{sim}$ como el monto de siniestros simulado de las pólizas con inicio de vigencia en el trimestre $i$ que fueron reportados $j$ trimestres posteriores al inicio de vigencia:
	
	\begin{equation}
    X_{i,j}^{sim}=R_{i,j}^{*}+X_{i,j}^{*} 
\label{eq8}    
\end{equation}
	
	Donde $R_{i,j}^{*}$ es el residual seleccionado en la muestra que corresponde a las pólizas con inicio de vigencia en el trimestre $i$ con trimestre de reporte $j$, y $X_{i,j}^{*}$ es el monto de siniestros ajustados de las pólizas con inicio de vigencia en el trimestre $i$ que fueron reportados $j$ trimestres posteriores al inicio de vigencia.
	
	La matriz queda de la siguiente manera:
	
	\begin{table}[H]
		\centering
		\caption{Matriz de Siniestros Acumulados Simulados}
		\begin{tabularx}{\linewidth}{ c|ccccccccc}
			\multirow{2}{4cm}{Trimestre de inicio de vigencia de la póliza}& \multicolumn{9}{c}{Trimestre en que se reportó el procedimiento} \\
			& 0  & 1 & 2 & $ \dots $ & $j$ & $\dots $ & k-2 & k-1 &  k \\
			\midrule
			1      &  $X_{1,0}^{sim}$ & $X_{1,1}^{sim}$ & $X_{1,2}^{sim}$ & $ \dots $ & $X_{1,j}^{sim}$ & $ \dots $ & $X_{1,k-2}^{sim}$ & $X_{1,k-1}^{sim}$ & $X_{1,k}^{sim}$ \\
			2      &  $X_{2,0}^{sim}$ & $X_{2,1}^{sim}$ & $X_{2,2}^{sim}$ & $ \dots $ & $X_{2,j}^{sim}$ & $ \dots $ & $X_{2,k-2}^{sim}$ & $X_{2,k-1}^{sim}$ & \\
			3      &  $X_{3,0}^{sim}$ & $X_{3,1}^{sim}$ & $X_{3,2}^{sim}$ & $ \dots $ & $X_{3,j}^{sim}$ & $ \dots $ & $X_{3,k-2}^{sim}$ & & \\
			4      &  $X_{4,0}^{sim}$ & $X_{4,1}^{sim}$ & $X_{4,2}^{sim}$ & $ \dots $ & $X_{4,j}^{sim}$ & $ \dots $ & & & \\
			:      & & & & & & & & &\\
			i      &  $X_{i,0}^{sim}$ & $X_{i,1}^{sim}$ & $X_{i,2}^{sim}$ & $ \dots $ & $X_{i,j}^{sim}$ & & & & \\
			:      & & & & & & & & & \\
			s-2      &  $X_{s-2,0}^{sim}$ & $X_{s-2,1}^{sim}$ & $X_{s-2,2}^{sim}$ & & & & & & \\
			s-1      &  $X_{s-1,0}^{sim}$ & $X_{s-1,1}^{sim}$ & & & & & & & \\
			s      &  $X_{s,0}^{sim}$ & & & & & & & & \\
		\end{tabularx}
	\end{table}

Ya que obtenemos esta matriz de siniestros simulada, generamos la matriz de siniestros acumulados simulados, obtenemos los factores de incremento simulados, estimamos los siniestros esperados simulados y calculamos los flujos de obligaciones futuras simuladas de riesgos en curso de la muestra $i$ ($RRC_{i}^{sim}$) mediante el proceso usado para la matriz original de siniestros.
	
	 
	
	Consideramos el mejor estimador de riesgos en curso ($BELR_{RRC}$), como el valor medio de las $n$ muestras de los flujos de obligaciones futuras simuladas de riesgos en curso.


	 

	
	\begin{equation}
		BELR_{RRC}^{}=\frac{\sum _{i=1}^{n}RRC_{i}^{sim}}{n}
		\label{eq9}
		\end{equation}
			


 
	
	Donde
	
	 
		\begin{itemize}
		\setlength\itemsep{-0.5em}
	\item $RRC_{i}^{sim}=$ i-ésima simulación de los flujos de obligaciones futuras de riesgos en curso.
	
	\item $n=$ número de simulaciones realizadas.
	\end{itemize}
	
\begin{comment}	
	 
	Para Salud Individual Dental, el $BELR_{RRC}$ se calculará como el producto de la prima de tarifa no devengada y el factor de siniestralidad última de mercado proporcionado por la Comisión Nacional de Seguros y Fianzas:


	 

$ $

 
	
	
	{\centering
		$BELR_{RRC}^{}=PTND_{} \cdot FS_{BEL}^{RRC}$
		\noindent
		
	}
	
	 

$ $

 
	
	Donde
	
	 
	
	$PTND_{}=$ Prima de tarifa no devengada.
\begin{comment}	
	$FS_{BEL}^{RRC}=$ Factor de Siniestralidad última con información de mercado.
	
	 

$ $

 

	El cálculo del $BELR_{RRC}$, se realizará de forma trimestral y se prorrateara con el vigor de la valuación del cierre de mes a fin de obtener la reserva de riesgos en curso.

\end{comment}
	
	 
	\section{Cálculo del Bel de Gastos de Administración}
	 
	El mejor estimador de la reserva de Gastos de Administración ($BELG_{ADM}$) es el monto integrado por la suma de los Gastos de Administración no devengados de cada uno de los asegurados en vigor.
	
	 
	
	Se determinará como la porción correspondiente a los Gastos de Administración de la Prima de Tarifa de los asegurados de que se trate multiplicada por el Factor de No Devengamiento ($F_{ND}$). 
\begin{comment}
Para Salud Individual Dental, se tomara el porcentaje de gasto de administración del mercado proporcionado por la Comisión Nacional de Seguros y Fianzas.
\end{comment}	
	 
	
	Sea $GELG_{ADM,ind}$, el monto de los gastos de administración no devengado de cada asegurado:
	
	 

\begin{comment}	
$
BELG_{ADM,ind}=\begin{cases}
PT \cdot G_{ADM} \cdot F_{ND}, & \text{$g \neq Salud  Dental  Individual$}.\\
PT \cdot \alpha_{i} \cdot F_{ND}, & \text{$g = Salud  Dental  Individual$}.
\end{cases}
$
\end{comment}

\begin{equation}
BELG_{ADM,ind}=	PT \cdot G_{ADM} \cdot F_{ND}
\label{eq10}
\end{equation}



	Donde
	 
		\begin{itemize}
		\setlength\itemsep{-0.5em}
    \item $PT=$ Prima de tarifa
    
    \item $G_{ADM}^{}=$ Gasto de Administración
	
	\item $F_{ND}^{}=$ Factor de no devengamiento
	\end{itemize}
	
\begin{comment}	
	$\alpha_{i}^{}=$ Porcentaje de gasto de administración con información de Mercado
\end{comment}	
	 


	Entonces el BEL para Gastos de Administración ($BELG_{ADM}$) es:
	

\begin{equation}
	BELG_{ADM}=\sum _{}^{}BEL_{ADM,ind}^{}
	\label{eq11}
	\end{equation}


 
	
	\section{Cálculo de la Prima de Riesgo No Devengada}
	 
	
	La Prima de Riesgo No Devengada corresponderá al valor de la prima de riesgo multiplicada por el factor de no devengamiento correspondiente a la porción de tiempo de vigencia no transcurrido.
	
	 
	
	Para el cálculo de la Prima de Riesgo No Devengada, se determinará para cada uno de los asegurados en vigor, la Prima de Riesgo, que corresponde al costo esperado de la siniestralidad y es la porción de la prima de tarifa que debe destinarse para el pago de las reclamaciones por concepto de siniestros.
	
	 
	
	Sea la Prima de Tarifa (PT):
	
	 
\begin{equation}
		{PT}_{}^{}=\frac{{PR}_{}^{}}{{1}_{}-{}G_{ADM}-{C}_{ADQ}-{U}_{}}
		\label{eq12}
\end{equation}
 
	
	Donde
	
	 
		\begin{itemize}
		\setlength\itemsep{-0.5em}
	\item $PR=$ Prima de Riesgo
	
	\item $G_{ADM}^{}=$ Gasto de Administración
	
	\item $C_{ADQ}^{}=$ Costo de Adquisición
	
	\item $U_{}^{}=$ Margen de Utilidad
	\end{itemize}
	 
 
	
	Entonces:
	
\begin{equation}
		{PR}_{}^{}={{PT}_{}\cdot(1-G_{ADM}-C_{ADQ}-U)}
		\label{eq13}
		\end{equation}


	Una vez determinada la Prima de Riesgo, se calculará la Prima de Riesgo no Devengada de cada uno de los asegurados como la Prima de Riesgo multiplicada por el Factor de No Devengamiento.
		
	El Factor de No Devengamiento $F_{ND}$ es el factor que se utiliza para calcular la porción de tiempo de vigencia no transcurrido por cada asegurado en vigor.
	
	Se define FND como el Factor de No Devengamiento, de tal forma que:
		
	\begin{equation}	
		F_{ND}=\begin{cases}
		0, & \text{si } FVal \geqslant FFin.\\
		\frac{FFin-FVal}{FFin-FIni}, & \text{si } FIni \leqslant FVal \leqslant FFin\\
		1, & \text{si } FIni \geqslant FVal
		\end{cases}
	\label{eq14}
		\end{equation}

 
	$ $
	
	Donde
	
	 
		\begin{itemize}
		\setlength\itemsep{-0.5em}
	\item $FIni=$ Fecha de Inicio de cobertura para el asegurado
	
	\item $FFin=$ Fecha de Fin de cobertura para el asegurado
	
	\item $FVal=$ Fecha de Valuación

\end{itemize}
 
	
	Entonces la Prima de Riesgo No Devengada ($PRND_{ind}$) es:
	

	
\begin{equation}
		{PRND}_{ind}^{}={{PR}_{ind}\cdot F_{ND}}
		\label{eq15}
\end{equation}

 
	
	Donde
	
	\begin{itemize}
	\setlength\itemsep{-0.5em}	 
	
	\item $F_{ND}=$ Factor de No Devengamiento de cada asegurado
	
	\item $PR_{ind}=$ Prima de Riesgo de cada asegurado 
	\end{itemize}
	 

 
	
	Para las pólizas emitidas anticipadamente que, al momento de la valuación, no han iniciado vigencia, la prima de riesgo no devengada se calculará como:
	
	 

\begin{equation}
		{PRND}_{ind}^{}={{PT}_{ind}-BELG_{ADM,ind}}
\label{eq16}
\end{equation}
 
	
	Donde
	
	 
	\begin{itemize}
	\setlength\itemsep{-0.5em}
	\item $PT_{ind}=$ Prima de Tarifa de cada asegurado
	
	\item $BELG_{ADM,ind}=$ Monto de los gastos de administración no devengado de cada asegurado
	\end{itemize}

	
	Sea $PRND_{}$ la prima de riesgo no devengada, esta se calculará como:
	
\begin{equation}
	PRND_{}={\sum _{}^{}PRND_{ind}^{}}
	\label{eq17}
\end{equation}

 
	Donde $PRND_{ind}=$ Prima de riesgo no devengada de cada asegurado
	
	
	
	\section{Cálculo del Factor de Distribución}
	
	 
	
	El Factor de Distribución permite prorratear el mejor estimador de riesgos en curso ($BELR_{RRC}$) obtenido entre cada asegurado. El factor se obtiene comparando la Prima de Riesgo No Devengada de cada grupo homogéneo ($PRND_{}$) calculada con el $BELR_{RRC}$ del mismo.
	
	 
	
	Sea $FD_{}$ el factor de distribución:
	
	 
\begin{equation}
		{FD}_{}^{}=\frac{{BELR}_{RRC}^{}}{{PRND}_{}}
		\label{eq18}
\end{equation}
 
	
	Donde
	
	 
		\begin{itemize}
		\setlength\itemsep{-0.5em}
		\item $BELR_{RRC}=$ Mejor Estimador de riesgos en curso
		
		\item $PRND_{}=$ Prima de Riesgo No Devengada
		
	 	\end{itemize}

$ $

 
	
	\section{Cálculo de la Reserva de Riesgos en Curso de cada Asegurado}
	
	 
	
	La Reserva de Riesgos en Curso de cada uno de los asegurados en vigor se calculará como la Prima de Riesgo No Devengada de cada asegurado multiplicada por el Factor de Distribución \begin{comment} y sumando el $BELG_{ADM,ind}$ así como el Margen de Riesgo prorrateado.
	\end{comment}
	 
	
	Sea $BELR_{RRC,ind}$ el mejor estimador de riesgos en curso individual:
	


	\begin{comment}	
		$
		BELR_{RRC,ind}=\begin{cases}
		PRND_{ind}\cdot FD_{}, & \text{$g \neq Salud Dental Individual$}\\
	
		PTND_{ind}\cdot FS_{BEL}^{RRC}, & \text{$g = Salud Dental Individual$}
		\end{cases}
		$
\end{comment}	

\begin{equation}
	BELR_{RRC,ind}=PRND_{ind}\cdot FD_{}
	\label{eq19}
\end{equation}

 
	
	Donde
	
		
	\begin{itemize}
	\setlength\itemsep{-0.5em}	
	\item $PRND_{ind}=$ Prima de Riesgo no Devengada de cada asegurado
	
	\item $FD_{}^{}=$ Factor de Distribución
	\end{itemize}
	
		\begin{comment}
	$PTND_{ind}=$ Prima de Tarifa no Devengada de cada asegurado
	

	$FS_{BEL}^{RRC}=$ Factor de Siniestralidad última con información de mercado
	\end{comment}
	
	 
	Entonces la Reserva de Riesgos en Curso de cada asegurado ($RRC_{ind}$) es:
	
\begin{equation}
		{RRC}_{ind}^{}={{BELR}_{RRC,ind}+BELG_{ADM,ind} \begin{comment} +MR_{RRC,ind}\end{comment}
		}
	\label{eq20}
		\end{equation}	
			
	Donde
	 
		\begin{itemize}
		\setlength\itemsep{-0.5em}
	\item $BELR_{RRC,ind}=$ Mejor estimador de riesgos en curso individual
	
	\item $BELG_{ADM,ind}^{}=$ Monto no devengado de Gastos de Administración de cada asegurado
	\end{itemize}
	
	\begin{comment}
	$MR_{RRC,ind}^{}=$ Margen de Riesgo de la reserva de riesgos en curso calculado anteriormente
	\end{comment}
		
	Por lo tanto, la Reserva de Riesgos en Curso ($RRC_{}$) se calculará como:
	
\begin{equation}
		RRC_{}={\sum _{}^{}RRC_{ind}^{}}
		\label{eq21}
	\end{equation}
		
	
	Donde $RRC_{ind}^{}=$ Reserva de Riesgos en Curso de cada asegurado
	


 

%------------------------------------------------------------------------------------------ %
\clearpage
\appendix 

\fancyhead[LE]{\scshape\thepage\hspace{1cm}\footnotesize\nouppercase{Apéndices}}
\fancyhead[RO]{\scshape\footnotesize\nouppercase{Apéndice \thechapter}\hspace{1cm}\normalsize\thepage}
\fancyhead[LO]{}
\fancyhead[RE]{}
\pagestyle{fancy}

\chapter{Metodología codificada en R }\label{desarrolloconstadm}

A continuación se presenta la Metodología descrita en este trabajo convertida a codigo de R. Se realiza con el fin de corroborar los resultados de la Metodología aplicada manualmente, así como facilitar las $n$ simulaciones necesarias.


Instalamos los paquetes necesarios de R si no los tenemos

\begin{lstlisting}
## Instalamos los paquetes necesarios de R si no los tenemos
install.packages("ChainLadder")
install.packages("psych")

## Referenciamos los paquetes instalados mediante las bibliotecas
library(ChainLadder)
library(psych)

## Función Remover para limpiar la consola
rm(list=ls())



### Definimos la funcion de Reservas RRC
SIMULACION_RRC <- function (MATRIX_INICIAL, n, archivo){
	
	### Definimos la dimension de los triángulos de residuales iniciales
	library(ChainLadder)
	m=dim(MATRIX_INICIAL)[1]
	
	
	### Inicia el cálculo
	
	## La siguiente sección corresponde a la plantilla de Excel
	
	## Nombramos la función cumulativa de triangulos incrementales
	## Estos triangulos son nuestra matriz de siniestros
	MATRIX_SIN_ACUM <- incr2cum(MATRIX_INICIAL)
	
	## Definimos los factores como indica el escrito
	# Usando la matriz de siniestros acumulados obtenemos los factores de incremento
	# fj, los cuales indican el incremento dado de un trimestre a otro:
	FACTORES_J <- sapply(1:(m-1), function(i){ sum(MATRIX_SIN_ACUM[c(1:(m-i)), i+1])/sum(MATRIX_SIN_ACUM[c(1:(m-i)), i]) } )
	
	## Mediante estos factores de incremento, se definen los Siniestros Esperados
	## para la vigencia i SEi como la estimación del monto de siniestros serán 
	## reportados para las pólizas con inicio de vigencia en el trimestre i:
	SIN_ESPERADOS <- cbind(MATRIX_SIN_ACUM, RVA = rep(0,m))
	for (j in 1:m){
		for(i in 1:m){
			if(is.na(SIN_ESPERADOS[i,j])){
				SIN_ESPERADOS[i,j] = SIN_ESPERADOS[i,(j-1)]*FACTORES_J[j-1]
			}
		}
	}
	
	# Definimos las obligaciones futuras iniciales de riesgos en curso RRC0 como la 
	# diferencia entre los Siniestros Estimados y los Siniestros Acumulados
	# observados:
	for (i in 1:m){
		SIN_ESPERADOS[i,(m+1)] = SIN_ESPERADOS[i,m]-SIN_ESPERADOS[i,(m-i+1)]
	}
	
	# Definimos Yi,j* como el monto ajustado de siniestros de las pólizas
	# con inicio de vigencia en el trimestre i reportados hasta el trimestre j
	# y con estos resultados obtenemos la matriz de siniestros acumulados ajustados
	MATRIX_ACUM_AJUST = MATRIX_SIN_ACUM
	for (i in 1:(m-1)){
		for(j in 1:(m-i)){
			MATRIX_ACUM_AJUST[i,(m-j-i+1)] = MATRIX_ACUM_AJUST[i,(m-j-i+2)]/FACTORES_J[m-i-j+1]
		}
	}
	
	# Y a partir de esta matriz de siniestros acumulados ajustados se obtiene una
	# nueva matriz de montos Xi,j*, que son los siniestros ajustados de las
	# pólizas con inicio de vigencia en el trimestre i que fueron reportados j 
	# trimestres posteriores al inicio de vigencia:
	MATRIX_SIN_AJ = cum2incr(MATRIX_ACUM_AJUST)
	
	# La diferencia del monto de siniestros observados (el monto original) y el 
	# monto de siniestros ajustado son los residuales brutos Ri,j:
	RESIDUAL = MATRIX_INICIAL - MATRIX_SIN_AJ
	
	## Definimos una nueva Matriz de n por m-1
	SIMULACION_RESERVA=c()
	
	## INICIAMOS LAS n SIMULACIONES
	for (k in 1:n){
		# DEFINIMOS UNA MUESTRA DE RESIDUALES
		MUESTRA_RESIDUALES = matrix(nrow=m, ncol=m)
		
		## Utilizamos el método de bootstrap, bajo el supuesto de que los residuales
		## brutos $R_{i,j}$ de la matriz provienen de la misma distribución y son 
		## independientes. Obtenemos el valor mínimo y máximo observado de cada columna
		## j como el intervalo de residuales observado.
		for(j in 1:m){
			# Definimos Ri,j* como el residual de la muestra de pólizas con 
			# inicio de vigencia en el trimestre i con trimestre de reporte j. 
			MUESTRA_RESIDUALES[,j] = t(runif(dim(RESIDUAL)[1],min=min(na.omit(RESIDUAL[,j])), max=max(na.omit(RESIDUAL[,j]))))
			##se elimina el rango de 10% de escenarios no observados
		}
		
		#	Obtenemos así una matriz de siniestros simulada al agregar el residual 
		# obtenido a cada monto de siniestros ajustados.
		#	Para esto definimos Xi,j{sim} como el monto de siniestros simulado de
		# las pólizas con inicio de vigencia en el trimestre i que fueron reportados 
		# j trimestres posteriores al inicio de vigencia:
		SINIESTRO_SIM = MUESTRA_RESIDUALES + MATRIX_SIN_AJ
		
		#generamos la matriz de siniestros simulada
		MATRIX_SIN_SIM <- incr2cum(SINIESTRO_SIM)
		
		## Repetimos el proceso original
		
		## Definimos los factores como indica el escrito
		# Usando la matriz de siniestros acumulados obtenemos los factores de incremento
		# fj, los cuales indican el incremento dado de un trimestre a otro:
		FACTOR_N_SIM <- sapply(1:(m-1), function(i){ sum(MATRIX_SIN_SIM[c(1:(m-i)), i+1])/sum(MATRIX_SIN_SIM[c(1:(m-i)), i]) } )
		
		## Mediante estos factores de incremento, se definen los Siniestros Esperados
		## para la vigencia i SEi como la estimación del monto de siniestros serán 
		## reportados para las pólizas con inicio de vigencia en el trimestre i:
		SIN_ESP_N_SIM <- cbind(MATRIX_SIN_SIM, RVA = rep(0,m))
		for (j in 1:m){
			for(i in 1:m){
				if(is.na(SIN_ESP_N_SIM[i,j])){
					SIN_ESP_N_SIM[i,j] = SIN_ESP_N_SIM[i,(j-1)]*FACTOR_N_SIM[j-1]
				}
			}
		}
		
		# Definimos las obligaciones futuras iniciales de riesgos en curso RRC0 como la 
		# diferencia entre los Siniestros Estimados y los Siniestros Acumulados
		# observados:
		for (i in 1:m){
			SIN_ESP_N_SIM[i,(m+1)] = SIN_ESP_N_SIM[i,m]-SIN_ESP_N_SIM[i,(m-i+1)]
		}
		
		## VALOR DE LA RESERVA
		SIMULACION_RESERVA[k]=sum(SIN_ESP_N_SIM[,m+1])
		
		
	}
	
	# Obligaciones Futuras Iniciales de Riesgos en Curso inicial
	RRC_INICIAL=sum(SIN_ESPERADOS[,m+1])
	
	# Media de las n simulaciones de Obligaciones Futuras Iniciales de Riesgos en Curso
	RRC_MEDIA=mean(SIMULACION_RESERVA)
	
	
	## Resultados tex de la funcion
	write.table (SIMULACION_RESERVA, file = paste("RRC_N_SIM_", archivo, ".txt", sep=""), quote = FALSE, sep = "\t", eol = "\n", dec = ".", row.names = FALSE, col.names = FALSE)
	print(paste("RRC_N_SIM_", archivo, ".txt", sep=""))
	
	write.table (c(RRC_INICIAL=RRC_INICIAL, media=RRC_MEDIA), file = paste("RRC_INICIAL_MEDIA", archivo, ".txt", sep=""), quote = FALSE, sep = "\t", eol = "\n", dec = ".", row.names = FALSE, col.names = FALSE)
	print(paste("RRC_INICIAL_MEDIA", archivo, ".txt", sep=""))
	
	
	print("FIN DE LA FUNCION")
	
	return(list(SIMULACION_RESERVA, c(RRC_INICIAL=RRC_INICIAL, media=RRC_MEDIA)))
	
}



TRIANGULO_SINIESTROS_CSV<-read.csv("D:/RESPALDO/Tesis Chris/Triangulo Siniestros Prueba.csv")

RESULTADO_RRC<-SIMULACION_RRC(TRIANGULO_SINIESTROS_CSV, 100000, "TRIANGULO_SINIESTROS_CSV")


\end{lstlisting}

	
	\pagestyle{empty}
\bibliography{library}
	
\end{document}
